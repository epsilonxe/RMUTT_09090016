\documentclass[a4paper]{article}
\usepackage{minted}
\usepackage{url, hyperref}
\usepackage{xcolor}
\usepackage{tcolorbox}
\tcbuselibrary{listings, minted, skins}
\usepackage{thaispec}

\title{Python Mini Projects}
\author{Ratthaprom Promkam, Dr.rer.nat}
\date{Revised: 24 September 2021}



\definecolor{inline}{RGB}{187,57,82}
\definecolor{bg}{RGB}{22,43,58}

\tcbset{listing engine=minted}

\newtcblisting{pylst}{listing only, minted language=python, minted style=paraiso-dark,
    colback=bg, enhanced, frame hidden, minted options={fontfamily=fdm, 
    fontsize=\footnotesize, tabsize=2, breaklines, autogobble}}

\setminted[python]{linenos, tabsize=4, breaklines, autogobble}

\begin{document}

\maketitle
เอกสารนี้ประกอบไปด้วยตัวอย่างโครงงานภาษาโปรแกรมไพธอนที่น่าสนใจ เหมาะกับผู้ที่เริ่มต้นเรียนรู้การพัฒนาซอฟท์แวร์คอมพิวเตอร์

\tableofcontents

\cleardoublepage
\section{Guessing Number}
โครงงานนี้เป็นการพัฒนาเกมในรูปแบบอย่างง่าย โดยให้ผู้เล่นทำการพยายามคาดเดาเลขที่โปรแกรมกำหนดไว้ล่วงหน้า เมื่อเริ่มต้นเกมโปรแกรมจะทำการสุ่มเลขมา 1 ตัวจากช่วงที่กำหนดไว้ (เช่น 1 - 10)  และผู้เล่นจะต้องทำการคาดเดาเลขไปจนกว่าจะตอบถูก เมื่อผู้เล่นตอบผิดในแต่ละครั้ง โปรแกรมจะต้องทำการแจ้งผู้เล่นว่าเลขที่ไม่ถูกต้องนั้นมีค่ามากไปหรือน้อยไปจากเลขที่ถูกต้อง

ในการพัฒนาโปรแกรมนี้ผู้พัฒนาอาจมีการใส่ลักษณะเสริมเพิ่มเติมเช่น จำนวนครั้งการเดาที่จำกัด การใส่คำสั่งที่ทำให้ผู้เล่นสามารถออกจากเกมได้แม้ว่ายังตอบไม่ถูกต้อง หรือการใส่คำสั่งพิเศษที่ใส่เข้าไปแล้วผู้เล่นสามารถรู้คำตอบที่ถูกต้องได้ทันที เป็นต้น

\subsection*{ตัวอย่างโปรแกรม}

ในโปรแกรมนี้เราสามารถอิมพอร์ตฟังก์ชันในไลบรารีมาตรฐานอย่าง \mintinline{python}{random}\footnote{ดูคำสั่งเพิ่มเติมของไลบรารีนี้ได้จาก \url{https://docs.python.org/3/library/random.html}} เพื่อใช้คำสั่งในการสุ่มเลขจากช่วงที่กำหนด เช่น

\begin{center}
\begin{tabular}{|l|p{0.7\linewidth}|}
\hline
ฟังก์ชัน & ผลลัพธ์  \\  
\hline

\mintinline{python}{random.random()} &
ส่งกลับค่าตัวเลขทศนิยมที่สุ่มมาจากช่วง $[0.0, 1.0)$ \\
\hline

\mintinline{python}{random.uniform(a, b)} &
ส่งกลับค่าตัวเลขทศนิยมที่สุ่มแบบเอกรูปมาจากช่วง [\mintinline{python}{a}, \mintinline{python}{b}] \\
\hline

\mintinline{python}{random.randint(a, b)} &
ส่งกลับค่าตัวเลขจำนวนเต็มมาจากช่วง [\mintinline{python}{a}, \mintinline{python}{b}] \\
\hline

\end{tabular}
\end{center}

\inputminted{python}{guessing_number.py}

\subsection*{ตัวอย่างผลลัพธ์}

\begin{minted}{console}
GUESSING NUMBER GAME
Try to guess my secret number from 1 - 100
Try to guess my number: 50
Your guessed nunber is too low
Try to guess my number: 75
Your guessed nunber is too low
Try to guess my number: 90
Your guessed nunber is too low
Try to guess my number: 95
Your guessed nunber is too low
Try to guess my number: 97
Your guess is correct!
Thanks for playing    
\end{minted}

\cleardoublepage
\section{Rock-Paper-Scissors}
โครงงานนี้เป็นการพัฒนาเกมเป่ายิ้งฉุบในรูปแบบอย่างง่าย โดยให้ผู้เล่นเลือกออกคำสั่ง: ค้อน, กรรไกร หรือกระดาษ อย่างใดอย่างหนึ่ง และโปรแกรมก็สุ่มเลือกคำสั่ง: ค้อน, กรรไกร หรือกระดาษ อย่างใดอย่างหนึ่งเช่นกัน โดยการแพ้ชนะจะใช้หลักเกณฑ์มาตรฐานคือ
\begin{itemize}
    \item ค้อนชนะกรรไกร
    \item กรรไกรชนะกระดาษ
    \item กระดาษชนะค้อน
    \item ถ้ายังไม่มีผู้ชนะ (ผู้เล่นออกคำสั่งตรงกับโปรแกรม) เกมจะกลับไปเริ่มต้นใหม่อีกครั้ง
\end{itemize}

\subsection*{ตัวอย่างผลลัพธ์}

\begin{minted}{console}
ROCK-PAPER-SCISSORS GAME
------------------------------
Enter your choice (R/P/S): R
Rock (You) vs Paper (Com)
--> You lose


ROCK-PAPER-SCISORS GAME
------------------------------
Enter your choice (R/P/S): S
Scissors (You) vs Paper (Com)
--> You win


ROCK-PAPER-SCISORS GAME
------------------------------
Enter your choice (R/P/S): R
Rock (You) vs Rock (Com)
--> Draw --> Play again
Enter your choice (R/P/S): S
Scissors (You) vs Paper (Com)
--> You win
\end{minted}

\cleardoublepage
\section{Dices Roll Simulator}
โครงงานนี้ได้จำลองสถานการณ์การโยนลูกเต๋า \mintinline{python}{n} ลูก โดยลูกเต๋าแต่ละลูกเป็นอิสระต่อกัน และลูกเต๋าแต่ละลูกจะออกได้แต้ม 1-6 เท่านั้น

\subsection*{ตัวอย่างผลลัพธ์}
\begin{minted}{console}
How many dices do you want to roll: 1
[1] Rolling ... 4
Total: 4

How many dices do you want to roll: 3
[1] Rolling ... 5
[2] Rolling ... 1
[3] Rolling ... 3
Total: 5 + 1 + 3 = 9

How many dices do you want to roll: 5
[1] Rolling ... 1
[2] Rolling ... 1
[3] Rolling ... 4
[4] Rolling ... 6
[5] Rolling ... 2
Total: 1 + 1 + 4 + 6 + 2 = 14
\end{minted}

\cleardoublepage
\section{Password Generator}
โครงงานนี้เป้าหมายคือต้องการให้ผู้ใช้สร้างรหัสผ่านที่มีความยาว \mintinline{python}{n} ตัวอักษร จากอักขระ \mintinline{python}{a-z}, \mintinline{python}{A-Z}, \mintinline{python}{0-9} และ อักขระสัญลักษณ์เพิ่มเติมจาก \mintinline{python}{'@$+-*/&<>^_'} โดยมีข้อบังคับคือ
\begin{itemize}
    \item มีอักขระอย่างน้อย 1 ตัวเป็นตัวอักขระจาก \mintinline{python}{A-Z}
    \item มีอักขระอย่างน้อย 1 ตัวเป็นตัวอักขระจาก \mintinline{python}{0-9}
    \item มีอักขระอย่างน้อย 1 ตัวเป็นตัวอักขระจาก \mintinline{python}{'@$+-*/&<>^_'}
\end{itemize}

\subsection*{ตัวอย่างผลลัพธ์}
\begin{minted}{console}
Enter the length of your password: 8
r59Tb$8s

Enter the length of your password: 16
PO1Tx0$8<r5+Tb@8p
\end{minted}

\cleardoublepage
\section{Hangman Game}
โครงงานนี้เป็นการพัฒนาเกมแฮงแมน โดยโปรแกรมจะสุ่มคำมา 1 คำจากคลังคำศัพท์ และแสดงผลเป็น \mintinline{python}{_} แทนจำนวนตัวอักษรในคำนั้น เช่น หากคำที่โปรแกรมเลือกมาคือ \mintinline{python}{'HELLO'} โปรแกรมก็จะแสดงเป็น \mintinline{python}{'_ _ _ _ _'} หลังจากนั้นโปรแกรมก็จะให้ผู้เล่นคาดเดาตัวอักษรที่หายไป
\begin{enumerate}
    \item หากผู้เล่นตอบถูก โปรแกรมจะเปลี่ยนการแสดงจาก \mintinline{python}{_} เป็นตัวอักษรที่ผู้เล่นตอบถูก เช่นถ้าผู้เล่นตอบตัวอักษร \mintinline{python}{'L'} จากคำว่า \mintinline{python}{'HELLO'}  ได้ถูกต้อง โปรแกรมก็จะเปลี่ยนการแสดงผลจาก  \mintinline{python}{'_ _ _ _'} เป็น \mintinline{python}{'_ _ L L _'}
    \item เกมจะจบลงเมื่อผู้เล่นตอบถูกทุกตัวอักษร
    \item หากผู้เล่นตอบผิด โปรแกรมจะบันทึกจำนวนครั้งที่ตอบผิดไว้ และถ้าตอบผิดเกินกว่าจำนวนที่โปรแกรมตั้งไว้ เกมจะจบลงทันที
\end{enumerate}

\subsection*{ตัวอย่างผลลัพธ์}
\begin{minted}{console}
Hangman Game

        _ _ _ _ _
        
Try to guess a character: A
Your guess is wrong --> Mistake = 1

Try to guess a character: O
Your guess is right 

        _ _ _ _ O
        
Try to guess a character: I
Your guess is wrong --> Mistake = 2

Try to guess a character: H
Your guess is right 

        H _ _ _ O

Try to guess a character: L
Your guess is right 

        H _ L L O
        
Try to guess a character: E
Your guess is right 

        H E L L O
        
Thanks for playing!
\end{minted}


\cleardoublepage
\section{Quadratic Polynomial Solver}
โครงงานนี้เป็นการพัฒนาซอฟท์แวร์ในการหาผลเฉลยของพหุนามอันดับสองที่อยู่ในรูปของสมการ
\[ ax^2 + bx + c = 0 \]
เมื่อ $a \neq 0$, $b$ และ $c$ เป็นจำนวนจริงใด ๆ
โดยผลเฉลยของสมการนี้สามารถหาได้จาก
\[ x = \frac{-b \pm \sqrt{b^2-4ac}}{2a} \]

\subsection*{ตัวอย่างผลลัพธ์}
\begin{minted}{console}
Quadratic Polynomial Solver
        ax^2 + bx + c = 0 

Enter parameters a, b and c:  1 5 10
Solutions to x^2+7x+10 = 0 are -2.0 and -5.0

Enter parameters a, b and c:  2 5 -3
Solutions to 2x^2+5x-3 = 0 are -3.0 and 0.5
\end{minted}

\cleardoublepage
\section{Fibonacci Sequence Generator}
ลำดับฟิโบนัชชีคือลำดับที่อยู่ในรูปของ
\[ 1, 1, 2, 3, 5, 8, 13, 21, 43, \ldots \]
โดยลำดับนี้มีรูปแบบทั่วไปคือ
\[ a_n = a_{n-1} + a_{n-2} \]
เมื่อ $a_1 = a_2 = 1$

โครงงานนี้มีจุดประสงค์เพื่อพัฒนาซอฟท์แวร์ในการหา $n$ พจน์แรกของลำดับฟิโบนัชชี

\subsection*{ตัวอย่างผลลัพธ์}
\begin{minted}[breaklines]{console}
Fibonacci Sequence Generator

Enter the number of terms you want: 10
1, 1, 2, 3, 5, 8, 13, 21, 34, 55, ...

Enter the number of terms you want: 20
1, 1, 2, 3, 5, 8, 13, 21, 34, 55, 89, 144, 233, 377, 610, 987, 1597, 2584, 4181, 6765, ... 

Enter the number of terms you want: 30
1, 1, 2, 3, 5, 8, 13, 21, 34, 55, 89, 144, 233, 377, 610, 987, 1597, 2584, 4181, 6765, 10946, 17711, 28657, 46368, 75025, 121393, 196418, 317811, 514229, 832040, ...
\end{minted}


 
  
\end{document}